\documentclass[a4paper,12pt]{article}
\usepackage{fancyhdr}
\usepackage{fancyheadings}
\usepackage[ngerman,german]{babel}
\usepackage{german}
\usepackage[utf8]{inputenc}
%\usepackage[latin1]{inputenc}
\usepackage[active]{srcltx}
\usepackage{algorithm}
\usepackage[noend]{algorithmic}
\usepackage{amsmath}
\usepackage{amssymb}
\usepackage{amsthm}
\usepackage{bbm}
\usepackage{enumerate}
\usepackage{graphicx}
\usepackage{ifthen}
\usepackage{listings}
\usepackage{struktex}
\usepackage{hyperref}
\usepackage{color}
\usepackage{subfigure}
\usepackage{xcolor,listings}

\definecolor{javaBlue}{RGB}{42,0.0,255}
\definecolor{javaGreen}{RGB}{63,127,95}
\definecolor{javaPurple}{RGB}{127,0,85}
\definecolor{javaRed}{RGB}{127,0,0}
\lstset{
	keywordstyle=\color{javaPurple},
	commentstyle=\color{javaGreen},
	stringstyle=\color{javaGreen},
	breaklines=true,
	breakautoindent=true, 
	postbreak=\space,  
	tabsize=2,  
	basicstyle=\ttfamily\footnotesize, 
	showspaces=false,       
	showstringspaces=false, 
	extendedchars=true,      
	backgroundcolor=\color{black!10},
	emph={@Autowired,@Controller,@RequestMapping,@ModelAttribute,@Service,@Scope,@Entity,@Table,@Column,@Id}, 
	emphstyle=\color{javaBlue},
	captionpos=b
}

%%%%%%%%%%%%%%%%%%%%%%%%%%%%%%%%%%%%%%%%%%%%%%%%%%%%%%
%%%%%%%%%%%%%% EDIT THIS PART %%%%%%%%%%%%%%%%%%%%%%%%
%%%%%%%%%%%%%%%%%%%%%%%%%%%%%%%%%%%%%%%%%%%%%%%%%%%%%%
\newcommand{\Fach}{Informatik AG}
\newcommand{\Name}{Prof. Dr. Jörg Hettel; NovaTec Consulting GmbH (C. Uldack, T. Jakoby)}
\newcommand{\Semester}{SS 17}
\newcommand{\Uebungsblatt}{2}
%%%%%%%%%%%%%%%%%%%%%%%%%%%%%%%%%%%%%%%%%%%%%%%%%%%%%%
%%%%%%%%%%%%%%%%%%%%%%%%%%%%%%%%%%%%%%%%%%%%%%%%%%%%%%

\setlength{\parindent}{0em}
\setlength{\parskip}{1em}
\topmargin -1.0cm
\oddsidemargin 0cm
\evensidemargin 0cm
\setlength{\textheight}{9.2in}
\setlength{\textwidth}{6.0in}

%%%%%%%%%%%%%%%
%% Aufgaben-COMMAND
\newcommand{\Aufgabe}[1]{
  {
  %\vspace*{0.5cm}
  \textsf{\textbf{Aufgabe #1}}
  %\vspace*{0.2cm}
  
  }
}

%% Aufgaben-COMMAND
\newcommand{\Head}[1]{
	{
	%	\vspace*{0.5cm}
		\textsf{\textbf{#1}}
	%	\vspace*{0.2cm}
		
	}
}
%%%%%%%%%%%%%%
\hypersetup{
    pdftitle={\Fach{}: Übungsblatt \Uebungsblatt{}},
    pdfauthor={\Name},
    pdfborder={0 0 0}
}

\title{Übungsblatt \Uebungsblatt{}}
\author{\Name{}}

\begin{document}
\thispagestyle{fancy}
\lhead{\sf \large \Fach{} \\ \small \Name{}}
\rhead{\sf \Semester{}}
\vspace*{0.2cm}
\begin{center}
\LARGE \sf \textbf{Übungsblatt \Uebungsblatt{}}
\end{center}
\vspace*{0.2cm}

%%%%%%%%%%%%%%%%%%%%%%%%%%%%%%%%%%%%%%%%%%%%%%%%%%%%%%
%% Insert your solutions here %%%%%%%%%%%%%%%%%%%%%%%%
%%%%%%%%%%%%%%%%%%%%%%%%%%%%%%%%%%%%%%%%%%%%%%%%%%%%%%

\Head{Vorwort}
Dieses Übungsblatt dient als Leitfaden mit dessen Hilfe ein auf dem A*-Algorithmus basierender MarioAI-Agent erstellt werden kann.

\Aufgabe 1
Erstellen Sie Klassen um Graphen in Java zu verwalten.
\begin{enumerate}
	\item erstellen Sie die Klasse \lstinline|Node|
	\item erstellen Sie eine \lstinline|Graph| Klasse, welche \lstinline|Nodes| enthalten
	\item erweitern Sie die \lstinline|Graph|-Klasse um Methoden, mit welchen es möglich ist, \lstinline|Nodes| zu traversieren.
\end{enumerate}	

\Aufgabe 2
Implementieren Sie den A* Algorithmus mit Hilfe der in Aufgabe 1 erstellten Klassen.

\Aufgabe 3
Erweitern Sie die \lstinline|Node| Klasse um Attribute, welche für MarioAI wichtig oder hilfreich sein könnten.

\Aufgabe 4
Erstellen Sie eine Methode welche alle plausiblen \lstinline|MarioInputs| zurückgibt.

\Aufgabe 5
Erstellen Sie eine Methode, welche anhand eines übergebenen \lstinline|LevelState|
\begin{enumerate}
	\item diesen klont
	\item einen simulierten Tick darauf ausführt
	\item die Koordinaten von Mario zurückgibt (Hierzu können Sie die Klasse \lstinline|de.novatec.marioai.Coords|, Mario selbst oder eine andere Möglichkeit entwickeln)
\end{enumerate}

\Aufgabe 6
Nutzen Sie die Methode aus Aufgabe 5 um für alle plausiblen \lstinline|MarioInputs Nodes| zu erstellen.

\Aufgabe 7
Erstellen Sie eine Methode um eine Heuristik für einen aus Aufgabe 6 erstellten \lstinline|Node| zu berechnen. (Tipp: Sie müssen sich zuerst eine Heuristik ausdenken) 

\Aufgabe 8
Fügen Sie die nun bestehenden Elemente zu einem A* Algorithmus für MarioAI zusammen.

\Aufgabe 9
Berechnen Sie in \lstinline|doAiLogic()| den besten Pfad für Mario.
Tipp: Ignorieren Sie vorerst Münzen, total Kills, etc. 
Versuchen Sie lediglich das Levelende zu erreichen.

Überlegen Sie, wie der errechnete Pfad verwendet werden könnte, um eine \lstinline|MarioInput|-Instanz zurückzuliefern 
Diese wird von \lstinline|doAiLogic()| als Rückgabewert verwendet.

\textsf{\textbf{Aufgabe 10}} \\ \\
Optimieren Sie ihren Algorithmus um möglichst viele Punkte zu erlangen.

%%%%%%%%%%%%%%%%%%%%%%%%%%%%%%%%%%%%%%%%%%%%%%%%%%%%%%
%%%%%%%%%%%%%%%%%%%%%%%%%%%%%%%%%%%%%%%%%%%%%%%%%%%%%%
\end{document}
